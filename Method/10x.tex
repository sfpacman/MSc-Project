\documentclass{article}
\usepackage[utf8]{inputenc}
\begin{document}

\section*{10X marker-based classification}
\paragraph{}
This approach employs a customized set of known marker from the literature to infer the identity of cell clusters  based on the observed cluster-specific genes of each clusters. Several types of maker gens can be used to infer a specific cell-type. For example, a cluster can be identified as CD4+ T cells , if the cluster has a strong expression of CD3D (T cell) and CD4(CD4+ Cell). \\ 
\par
\begin{tabular}{ |l|c |r| }
\hline
Cell type&\multicolumn{2}{|c|}{Gene Marker} \\ \hline
myeloid cells&S100A8&S100A9 \\ \hline
B cells	&CD79A&\\ \hline	
NK cells&NKG7&\\	\hline
dendritic cells&FCER1A&\\ \hline	
T cells&CD3D&\\	\hline
memory T cells&CCR10&\\	\hline
regulatory T cells&TNFRSF18&\\ \hline	
CD8+&CD8A&\\	\hline
CD4+&CD4&\\	\hline
naive T-cell&ID3& \\	\hline
Activated cytotoxic T cells&NKG7*& \\ 	
\hline
\end{tabular}
\paragraph{}
* in conjunction with T cell gene marker: CD3D 
\section*{10X correlation with 11 purified PBMC populations}
\paragraph{}
This reference-based approach is based on the gene expression profile of 11 purified PBMC populations.
\paragraph{Selection of 11 purified PBMC populations} ~\\
To ensure purity of each population, FACS analyses is first carried out followed by cluster analysis of the gene expression (PCA and tSNE). If there is more than one cluster, genes markers are used for cell selection. For quality control, other genes markers are also used to check the identity of each population \\
\par
\begin{tabular}{ |l| r| }
\hline
CD19+ B&CD74,CD27 \\ \hline
CD14+ Monocyte&FTL* \\ \hline       
Dendritic&CLEC9A*,CD1C \\ \hline     
CD34+&CD34*\\ \hline
CD56+ NK&CD3D,NKG7 \\ \hline
CD4+/CD25 T Reg& NKG7 \\ \hline
CD4+/CD45RA+/CD25- Naive T&NKG7 \\ \hline
CD4+/CD45RO+ Memory&NKG7 \\ \hline
CD4+ T Helper2&NKG7 \\ \hline
CD8+/CD45RA+ Naive Cytotoxic& NKG7 \\ \hline 
CD8+ Cytotoxic T&CD8A,CD3D\\ \hline
\end{tabular}
\par
\paragraph{}
* Used for cell selection because of the multiple clusters in the population
\paragraph{Cell classification by using expression profile } ~\\
Each reference population is first normalized and downsampled to about 16k confidently mapped reads per cell to ensure consistency among the populations.Then, for each population, the average expression of genes of all cells is then calculated to form the expression profile. Spearman’s correlation is then calculated between each cells and each expression profile. The identity of Cell will then be assigned to the purified population with the highest correlation. Because some population are overlapped, such as CD4+ T helper and other CD4+ cells. Therefore, readjustment are required for the cells that has highest score and second highest score for those type of cells.     
\paragraph{Antibody used for FACS} ~\\
\begin{tabular}{ |l| c|r| }
\hline
CD4+/CD45RO+  T memory&CD4-FITC&CD45RO-PE\\ \hline
CD19+ B&CD19-FITC&\\ \hline
CD14+ Monocyte&CD14-FITC&\\ \hline
Dendritic&N/A&\\ \hline
CD56+ NK&CD56-FITC&\\ \hline
CD34+&CD34-FITC&\\ \hline
CD4+/CD25 T Reg&CD4-FITC&CD25-PE\\ \hline
CD4+/CD45RA+/CD25- Naive T&CD4-FITC&CD45RO-PE\\ \hline
CD4+ T Helper2&CD4-FITC&\\ \hline
CD8+/CD45RA+ Naive Cytotoxic&CD8-FITC&CD45RA-PE\\ \hline
CD8+ Cytotoxic T&CD8-FITC&\\ \hline
\end{tabular}

\end{document}
